\documentclass[]{article}
\usepackage{lmodern}
\usepackage{amssymb,amsmath}
\usepackage{ifxetex,ifluatex}
\usepackage{fixltx2e} % provides \textsubscript
\ifnum 0\ifxetex 1\fi\ifluatex 1\fi=0 % if pdftex
  \usepackage[T1]{fontenc}
  \usepackage[utf8]{inputenc}
\else % if luatex or xelatex
  \ifxetex
    \usepackage{mathspec}
    \usepackage{xltxtra,xunicode}
  \else
    \usepackage{fontspec}
  \fi
  \defaultfontfeatures{Mapping=tex-text,Scale=MatchLowercase}
  \newcommand{\euro}{€}
\fi
% use upquote if available, for straight quotes in verbatim environments
\IfFileExists{upquote.sty}{\usepackage{upquote}}{}
% use microtype if available
\IfFileExists{microtype.sty}{%
\usepackage{microtype}
\UseMicrotypeSet[protrusion]{basicmath} % disable protrusion for tt fonts
}{}
\usepackage[margin=1in]{geometry}
\usepackage{color}
\usepackage{fancyvrb}
\newcommand{\VerbBar}{|}
\newcommand{\VERB}{\Verb[commandchars=\\\{\}]}
\DefineVerbatimEnvironment{Highlighting}{Verbatim}{commandchars=\\\{\}}
% Add ',fontsize=\small' for more characters per line
\usepackage{framed}
\definecolor{shadecolor}{RGB}{248,248,248}
\newenvironment{Shaded}{\begin{snugshade}}{\end{snugshade}}
\newcommand{\KeywordTok}[1]{\textcolor[rgb]{0.13,0.29,0.53}{\textbf{{#1}}}}
\newcommand{\DataTypeTok}[1]{\textcolor[rgb]{0.13,0.29,0.53}{{#1}}}
\newcommand{\DecValTok}[1]{\textcolor[rgb]{0.00,0.00,0.81}{{#1}}}
\newcommand{\BaseNTok}[1]{\textcolor[rgb]{0.00,0.00,0.81}{{#1}}}
\newcommand{\FloatTok}[1]{\textcolor[rgb]{0.00,0.00,0.81}{{#1}}}
\newcommand{\CharTok}[1]{\textcolor[rgb]{0.31,0.60,0.02}{{#1}}}
\newcommand{\StringTok}[1]{\textcolor[rgb]{0.31,0.60,0.02}{{#1}}}
\newcommand{\CommentTok}[1]{\textcolor[rgb]{0.56,0.35,0.01}{\textit{{#1}}}}
\newcommand{\OtherTok}[1]{\textcolor[rgb]{0.56,0.35,0.01}{{#1}}}
\newcommand{\AlertTok}[1]{\textcolor[rgb]{0.94,0.16,0.16}{{#1}}}
\newcommand{\FunctionTok}[1]{\textcolor[rgb]{0.00,0.00,0.00}{{#1}}}
\newcommand{\RegionMarkerTok}[1]{{#1}}
\newcommand{\ErrorTok}[1]{\textbf{{#1}}}
\newcommand{\NormalTok}[1]{{#1}}
\usepackage{graphicx}
\makeatletter
\def\maxwidth{\ifdim\Gin@nat@width>\linewidth\linewidth\else\Gin@nat@width\fi}
\def\maxheight{\ifdim\Gin@nat@height>\textheight\textheight\else\Gin@nat@height\fi}
\makeatother
% Scale images if necessary, so that they will not overflow the page
% margins by default, and it is still possible to overwrite the defaults
% using explicit options in \includegraphics[width, height, ...]{}
\setkeys{Gin}{width=\maxwidth,height=\maxheight,keepaspectratio}
\ifxetex
  \usepackage[setpagesize=false, % page size defined by xetex
              unicode=false, % unicode breaks when used with xetex
              xetex]{hyperref}
\else
  \usepackage[unicode=true]{hyperref}
\fi
\hypersetup{breaklinks=true,
            bookmarks=true,
            pdfauthor={Hye Soo Choi},
            pdftitle={Homework7 Report},
            colorlinks=true,
            citecolor=blue,
            urlcolor=blue,
            linkcolor=magenta,
            pdfborder={0 0 0}}
\urlstyle{same}  % don't use monospace font for urls
\setlength{\parindent}{0pt}
\setlength{\parskip}{6pt plus 2pt minus 1pt}
\setlength{\emergencystretch}{3em}  % prevent overfull lines
\setcounter{secnumdepth}{0}

%%% Use protect on footnotes to avoid problems with footnotes in titles
\let\rmarkdownfootnote\footnote%
\def\footnote{\protect\rmarkdownfootnote}

%%% Change title format to be more compact
\usepackage{titling}

% Create subtitle command for use in maketitle
\newcommand{\subtitle}[1]{
  \posttitle{
    \begin{center}\large#1\end{center}
    }
}

\setlength{\droptitle}{-2em}
  \title{Homework7 Report}
  \pretitle{\vspace{\droptitle}\centering\huge}
  \posttitle{\par}
  \author{Hye Soo Choi}
  \preauthor{\centering\large\emph}
  \postauthor{\par}
  \predate{\centering\large\emph}
  \postdate{\par}
  \date{August 2, 2015}

\usepackage{graphicx}


\begin{document}

\maketitle


\begin{center}\rule{0.5\linewidth}{\linethickness}\end{center}

\subsection{Data source}\label{data-source}

In this study, we analyzed the data about storms that hit the world in
2010, in terms of location, speed, pressure, date of occurence, and
duration. We used the storm data of 2010
\href{Year.2010.ibtracs_wmo.v03r06.cxml}{(Year.2010.ibtracs\_wmo.v03r06.cxml)},
which can be obtained at the \textbf{International Best Track Archive
for Climate Stewardship (IBTrACS)}
\href{ftp://eclipse.ncdc.noaa.gov/pub/ibtracs/v03r06/wmo/cxml/year}{(\url{ftp://eclipse.ncdc.noaa.gov/pub/ibtracs/v03r06/wmo/cxml/year})}.

\begin{center}\rule{0.5\linewidth}{\linethickness}\end{center}

\subsection{Data cleaning}\label{data-cleaning}

A cleaned data.frame, \emph{storms}, was created by mainly using
\emph{XPATH} expressions from \textbf{XML} package, to extract various
pieces of data. After extracting the variables of interest and changing
the variable formats, the final data include the following variables:

\begin{itemize}
\itemsep1pt\parskip0pt\parsep0pt
\item
  name: name of the storm (e.g.~ANJA)
\item
  date: date (e.g.~2009-11-13)
\item
  time: time (e.g.~06:00:00)
\item
  latitude: latitude (e.g. -9.50)
\item
  lat\_deg: latitude degrees (e.g. ``N'')
\item
  longitude: longitude (e.g.~72.50)
\item
  lon\_deg``: longitude degrees (e.g. ``E'')
\item
  press: pressure (e.g.~1006.0)
\item
  wind: wind speed (e.g.~0.0)
\end{itemize}

The following displays the first 6 observations in the data.

\begin{Shaded}
\begin{Highlighting}[]
\KeywordTok{head}\NormalTok{(storms, }\DecValTok{6}\NormalTok{)}
\end{Highlighting}
\end{Shaded}

\begin{verbatim}
##   name       date     time latitude lat_deg longitude lon_deg press wind
## 1 ANJA 2009-11-13 06:00:00     -9.5       N      72.5       E  1006    0
## 2 ANJA 2009-11-13 12:00:00    -10.2       N      71.9       E  1004    0
## 3 ANJA 2009-11-13 18:00:00    -11.1       N      71.4       E  1002   25
## 4 ANJA 2009-11-14 00:00:00    -11.9       N      71.1       E   999   28
## 5 ANJA 2009-11-14 06:00:00    -12.5       N      70.9       E   996   33
## 6 ANJA 2009-11-14 12:00:00    -12.8       N      70.7       E   992   40
\end{verbatim}

\begin{center}\rule{0.5\linewidth}{\linethickness}\end{center}

\subsection{Data analysis}\label{data-analysis}

An exploratory analysis was performed with respect to the pressure, wind
speed, and duration of storms. We first examine the descriptive
statistics of the variables. The following table presents the summary
statistics of the three variables.

\begin{table}[ht]
\centering
\begin{tabular}{rrrrrrrr}
  \hline
 & mean & sd & min & max & 25\% & 50\% & 75\% \\ 
  \hline
pressure & 987.94 & 60.12 & 0.00 & 1016.00 & 985.00 & 998.00 & 1004.00 \\ 
  speed & 42.33 & 25.88 & 0.00 & 140.00 & 25.00 & 35.00 & 55.00 \\ 
  duration & 7.31 & 4.19 & 2.00 & 24.00 & 4.00 & 7.00 & 9.00 \\ 
   \hline
\end{tabular}
\caption{summary statistics of pressure, speed, duration} 
\end{table}

To understand the distribution, we also produced box plots and histogram
for each variable.

\includegraphics{Report_files/figure-latex/unnamed-chunk-6-1.pdf}
\includegraphics{Report_files/figure-latex/unnamed-chunk-6-2.pdf}
\includegraphics{Report_files/figure-latex/unnamed-chunk-6-3.pdf}

We can observe that the pressure is skewed to left, and the wind speed
and duration are skewed to left.

\begin{center}\rule{0.5\linewidth}{\linethickness}\end{center}

In addition, we also found several informative factors of the storms:

\begin{itemize}
\item
  Total number of storms in 2010: 71
\item
  Number of storms with winds \textgreater{}= 35 knots (tropical
  storms): 74
\item
  Number of storms with winds \textgreater{}= 64 knots (hurricanes): 40
\item
  Number of storms with winds \textgreater{}= 96 knots (major
  hurricanes) : 16
\item
  Number of storms on the north hemisphere: 52
\item
  Number of storms on the south hemisphere: 25
\item
  Frequency table of data points per month (month in words) in the
  world, northern hemisphere, and southern hemisphere:
\end{itemize}

\begin{verbatim}
## \begin{table}[ht]
## \centering
## \begin{tabular}{rrrrrrrrrrrrr}
##   \hline
##  & Jan & Feb & Mar & Apr & May & Jun & Jul & Aug & Sep & Oct & Nov & Dec \\ 
##   \hline
## World & 142 & 173 & 212 &  80 &  96 & 201 & 108 & 305 & 519 & 240 & 125 & 225 \\ 
##   Northern & 122 &  60 & 171 &  80 &  87 &  50 &  53 &  98 & 181 & 111 &  84 & 202 \\ 
##   Southern &  20 & 113 &  41 &   0 &   9 & 151 &  55 & 207 & 338 & 129 &  41 &  23 \\ 
##    \hline
## \end{tabular}
## \caption{Frequency table of data points per month in the world, hemisphere, southern hemisphere} 
## \end{table}
\end{verbatim}

\begin{itemize}
\item
  Name of storm that lasted most days : TWO
\item
  Name of storm with maximum wind speed and the maximum speed value:
  BLAS and 140
\item
  Name of storm with minimum pressure and the maximum pressure value:
  HUBERT, PHET and 0
\end{itemize}

\begin{center}\rule{0.5\linewidth}{\linethickness}\end{center}

\subsection{Data visualization}\label{data-visualization}

For the visualization of the data, we produced a world map with spatial
distribution of storm activity in 2010. The color of data points
reflects the intensity of each storm activity; the faster wind speed
was, the darker blue a point is colored on the map.

\includegraphics[width=500px,height=200px]{Report_files/figure-latex/unnamed-chunk-15-1}

\end{document}
